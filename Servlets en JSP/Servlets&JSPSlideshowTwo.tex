\documentclass{beamer}

\usepackage{verbatim}

\usetheme{CambridgeUS}

\usecolortheme{dolphin}

% De kleur van de url links bepalen
\definecolor{links}{HTML}{FF4444}
\hypersetup{colorlinks,linkcolor=,urlcolor=links}

%\setbeamertemplate{footline}[page number]

\setbeamertemplate{navigation symbols}{}

\title{Servlets en JSP}
\subtitle{deel twee}
\author{Java Cursisten}
\institute{INTEC Brussel}
\date{\today}

\begin{document}

\begin{frame}

\titlepage

\end{frame}


\begin{frame}

\frametitle{Overzicht}
{\Huge \tableofcontents}

\end{frame}



\section{JavaServer Pages}

\begin{frame}

\frametitle{Java en HTML team up... JSP}

{\Large Een JSP pagina lijkt op een HTML pagina.\\~\\

Maar je kan er wel Java en Java-gerelateerde 
zaken in kwijt.\\~\\

Het is alsof je variabelen in HTML zet.}


\end{frame}


\begin{frame}

\frametitle{JavaServer Pages bestaansredenen}

{\LARGE \begin{enumerate}
  \item Niet alle webontwerpers (HTML pagina ontwerpers) kunnen met Java overweg
  \item HTML formatten in een string is lelijke code en moeizaam
\end{enumerate}}

\end{frame}


\begin{frame}

\frametitle{Een JSP maken}

{\Large \begin{enumerate}
  \item Rechtermuisknop op WebContent - New - JSP File
  \item tik 'welkom' bij File name 
  \item Finish
  \item Voeg de code op de volgende slides toe
\end{enumerate}}

\end{frame}


\begin{frame}[fragile]

\begin{verbatim}

<%-- Een eerste JSP --%>
<%@ page contentType="text/html" pageEncoding="UTF-8" 
session="false"%>
<%@ page import = "java.io.PrintWriter" %@>
<%@ page import = "java.util.Calendar" %@>
<!doctype html>
<html lang="nl">
  <head>
    <title>Eerste JSP</title>
  </head>
  <body>
    <h1>
    
\end{verbatim}

\end{frame}


\begin{frame}[fragile]

\begin{verbatim}

<%
    Calendar calendar = Calendar.getInstance();
    int uur = calendar.get(Calendar.HOUR_OF_DAY);
    if (uur >= 6 && uur < 12) {
       out.print("Goede morgen");
    } else if (uur >= 12 && uur < 18) {
       out.print("Goede middag");
    } else {
       out.print("Goede avond");
    }
    %>
    </h1>
  </body>
</html>

\end{verbatim}

\end{frame}


\begin{frame}

\frametitle {JSP en Servlets team up... awesome web app}

{\Large \begin{enumerate}
  \item Een request wordt verwerkt door een servlet in de method 'doGet' of 'doPost'. Hier komt enkel Java code in te staan.
  \item Daarna wordt diezelfde request doorgegeven aan een JSP (to forward). Hier komt al de HTML in te staan.
\end{enumerate}}

\end{frame}


\begin{frame}[fragile]

\frametitle {To forward a request to a JSP}

{\Large Hier voor heb je een object nodig van de class 'RequestDispatcher'\\~\\
Dit object kan je verkrijgen via de methode 'getRequestDispatcher' van de request die je als parameter meekrijgt in de 'doGet' en 'doPost' methodes van de servlet.\\~\\
\begin{verbatim}RequestDispatcher dispatcher = 
request.getRequestDispatcher("Welkom.jsp"); \end{verbatim}
}

\end{frame}


\begin{frame}[fragile]

\frametitle {To forward a request to a JSP}

{\Large Het feitelijk doorgeven doe je met de methode 'forward' van de RequestDispatcher\\~\\

\begin{verbatim} dispatcher.forward(request, response) \end{verbatim}
}

\end{frame}


\begin{frame}[fragile]

\frametitle {Een waarde doorgeven van een servlet naar een JSP}

{\Large Het request object heeft hiervoor een methode.\\

\begin{verbatim}request.setAttribute
("De naam van het attribuut","Het attribuut");\end{verbatim}

Dit kan je meerdere keren doen voor meerdere waarden.}

\end{frame}


\begin{frame}[fragile]

\frametitle {De doorgegeven waarde lezen in een JSP}

{\Large Hiervoor gebruik je een miniprogrammeertaal, \\Expression Language (EL)\\~\\

Maandag meer !\\~\\

Vandaag enkel de syntax om de waarde van een attribuut te lezen

\begin{verbatim} ${naamVanHetRequestAttribuut} \end{verbatim}

}

\end{frame}


\begin{frame}

\frametitle{Concreet}

{\Large Fork de repository 'ServletsEnJSP' repository op
\href{https://github.com/VanbockryckInstructeur?tab=repositories}{GitHub}}

\end{frame}

\end{document}