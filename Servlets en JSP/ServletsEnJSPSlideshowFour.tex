\documentclass{beamer}

\usepackage{verbatim}

\usetheme{CambridgeUS}

\usecolortheme{dolphin}

%\setbeamertemplate{footline}[page number]

\setbeamertemplate{navigation symbols}{}

\title{Servlets en JSP}
\subtitle{deel vier}
\author{Java Cursisten}
\institute{INTEC Brussel}
\date{\today}

\begin{document}

\begin{frame}

\titlepage

\end{frame}

\begin{frame}

\frametitle{Overzicht}
{\LARGE \tableofcontents}

\end{frame}


\section{JavaServer Pages Standard Tag Library}


\begin{frame}

\frametitle{Wat is JSTL}

Om te vermijden dat je alsnog Java code zou gebruiken in JSP biedt JSTL een
reeks tags waarmee je programmatie logica kunt implementeren in een JSP.\\~\\

Dit maakt \textbf{maintenance} makkelijker en zorgt voor een duidelijkere splitsing tussen
Java code en JSP code.

\end{frame}


\begin{frame}

\frametitle{Maintenance?}

\textit{Software maintenance} gebeurt na de initi\"ele deployment van het product.
Het omvat het corrigeren van fouten, het aanpassen van functionaliteiten en het 
toevoegen van nieuwe functionaliteiten. Het succes van een software product hangt
voor een groot deel af van de \textbf{'maintainability'}. 

\end{frame}


\begin{frame}[fragile]

\frametitle{JSTL toepassen in JSP}

Allereerst moet je duidelijk maken dat je JSTL wilt gebruiken in je JSP met de
volgende page directive.

\begin{verbatim} 

<%@ taglib uri="http://java.sun.com/jsp/jstl/core" 
prefix="c" %>

\end{verbatim}

\end{frame}





\end{document}